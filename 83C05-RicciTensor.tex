\documentclass[12pt]{article}
\usepackage{pmmeta}
\pmcanonicalname{RicciTensor}
\pmcreated{2013-03-22 15:02:38}
\pmmodified{2013-03-22 15:02:38}
\pmowner{rmilson}{146}
\pmmodifier{rmilson}{146}
\pmtitle{Ricci tensor}
\pmrecord{9}{36758}
\pmprivacy{1}
\pmauthor{rmilson}{146}
\pmtype{Definition}
\pmcomment{trigger rebuild}
\pmclassification{msc}{83C05}
\pmdefines{scalar curvature}
\pmdefines{Einstein tensor}
\pmdefines{ricci scalar}
\pmdefines{Weyl tensor}
\pmdefines{Weyl curvature tensor}

\endmetadata

% this is the default PlanetMath preamble.  as your knowledge
% of TeX increases, you will probably want to edit this, but
% it should be fine as is for beginners.

% almost certainly you want these
\usepackage{amssymb}
\usepackage{amsmath}
\usepackage{amsfonts}

\newcommand{\Ric}{\operatorname{Ric}}
\begin{document}
\paragraph{Definition.}
The \emph{Ricci curvature tensor} is a rank $2$, symmetric tensor that
arises naturally in pseudo-Riemannian geometry.  Let $(M,g_{ij})$ be a
smooth, $n$-dimensional pseudo-Riemannian manifold, and let
$R^i{}_{jkl}$ denote the corresponding Riemann curvature tensor.  The
Ricci tensor $R_{ij}$ is commonly defined as the following contraction
of the full curvature tensor:
\[R_{ij} = R^k{}_{ikj}.
\] 
The index symmetry of $R_{ij}$, so defined, follows from the symmetry
properties of the Riemann curvature.  To wit,
\[ R_{ij} = R^k{}_{ikj} = R_{ki}{}^k{}_j = R^k{}_{jki} = R_{ji}.\]
It is also convenient to regard the Ricci tensor as a symmetric bilinear
form.  To that end for vector-fields  $X,Y$ we will write
\[ \Ric(X,Y) = X^i Y^j R_{ij}.\]
\paragraph{Related objects.}
Contracting the Ricci tensor, we obtain an important scalar invariant
\[R=R^i{}_i,\] called the scalar curvature, and sometimes also called
the Ricci scalar.  Closely related to the Ricci tensor is the tensor
\[G_{ij} = R_{ij} - \frac{1}{2} R\, g_{ij},\] called the \emph{Einstein
tensor}.  The Einstein tensor is also known as the trace-reversed Ricci
tensor owing to the fact that
\[  G^i{}_i = - R. \]
Another related tensor is
\[S_{ij} = R_{ij} - \frac{1}{n} R\, g_{ij}.\]
This is called 
the
trace-free Ricci tensor, owing to the fact that the above definition
implies that
\[ S^i{}_i=0.\]



\paragraph{Geometric interpretation.}
In Riemannian geometry, the Ricci tensor represents the average value
of the sectional curvature along a particular direction.  
Let 
\[ K_x(u,v) = \frac{R_x(u,v,v,u)}{g_x(u,u) g_x(v,v) - g_x(u,v)^2}
\]
denote the sectional curvature of $M$ along the plane spanned by
vectors $u,v\in T_x M$. Fix a point $x\in M$ and a tangent vector
$v\in T_xM$, and let
\[
S_x(v)=\{ u\in T_xM \colon g_x(u,u) = 1,\; g_x(u,v)=0 \}
\] denote the $n-2$ dimensional
sphere of those unit vectors at $x$ that are perpendicular to $v$.
 Let $\mu_x$ denote the natural
$(n-2)$-dimensional volume measure on $T_xM$, normalized so that
\[ \int_{S_x(v)} \mu_x = 1.\]
In this way, the quantity
\[ \int_{S_x(v)}\!\! K_x(\cdot,v) \mu_x, \]
describes the average value of the sectional curvature for all planes
in $T_x M$ that contain $v$.  It is possible to show that
\[ \Ric_x(v,v)= (1-n)\int_{S_x(v)}\!\! K_x(\cdot,v) \mu_x,\]
thereby giving us the desired geometric interpretation.


\paragraph{Decomposition of the curvature tensor.}
For $n\geq 3$, the Ricci tensor can be characterized in terms of the
decomposition of the full curvature tensor into three covariantly
defined summands, namely
\begin{align*}
  F_{ijkl} &= \tfrac{1}{n-2} \left( S_{jl}\, g_{ik}+S_{ik}\,
    g_{jl}-S_{il}\, g_{jk}-S_{jk}\, g_{il}\right),\\
  E_{ijkl} &= \tfrac{1}{n(n-1)}R \left( g_{jl}\,g_{ik} -
    g_{il}\,g_{jk}\right),\\
    W_{ijkl} &= R_{ijkl}-F_{ijkl}-E_{ijkl}.
\end{align*}
The $W_{ijkl}$ is called the \emph{Weyl curvature tensor}. It is
the conformally invariant, trace-free part of the curvature tensor.
Indeed, with the above definitions, we have
\[ W^k{}_{ikj}=0.\] The $E_{ijkl}$ and $F_{ijkl}$ correspond to the
trace-free part of the Ricci curvature tensor, and to the Ricci
scalar.  Indeed, we can recover $S_{ij}$ and $R$ from $E_{ijkl}$ and
$F_{ijkl}$ as follows:
\begin{align*}
  S_{ij} &=  F^k{}_{ikj},\\
  E^{ij}{}_{ij} &= R.
\end{align*}

\paragraph{Relativity.}  The Ricci tensor also plays an important role
in the theory of general relativity.  In this keystone application,
$M$ is a 4-dimensional pseudo-Riemannian manifold with signature
$(3,1)$.  The Einstein field equations assert that the energy-momentum
tensor is proportional to the Einstein tensor.  In particular, the
equation
\[ R_{ij}=0 \]
is the field equation for a vacuum space-time.  In geometry, a
pseudo-Riemannian manifold that satisfies this equation is called
Ricci-flat.  It is possible to prove that a manifold is Ricci flat if
and only if locally, the manifold, is conformally equivalent to flat space.
%%%%%
%%%%%
\end{document}
