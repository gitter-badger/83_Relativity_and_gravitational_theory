\documentclass[12pt]{article}
\usepackage{pmmeta}
\pmcanonicalname{MetricSuperfields}
\pmcreated{2013-03-22 18:19:08}
\pmmodified{2013-03-22 18:19:08}
\pmowner{bci1}{20947}
\pmmodifier{bci1}{20947}
\pmtitle{metric superfields}
\pmrecord{9}{40945}
\pmprivacy{1}
\pmauthor{bci1}{20947}
\pmtype{Topic}
\pmcomment{trigger rebuild}
\pmclassification{msc}{83E50}
\pmclassification{msc}{83C45}
\pmsynonym{supergravity fields}{MetricSuperfields}
%\pmkeywords{supergravity}
%\pmkeywords{quantum gravity}
%\pmkeywords{superspace}
%\pmkeywords{relativistic QFT and quantum field theories}
\pmrelated{SuperfieldsSuperspace}
\pmrelated{SpinNetworksAndSpinFoams}
\pmdefines{supergravity field}

\endmetadata

% this is the default PlanetMath preamble.  as your knowledge
% of TeX increases, you will probably want to edit this, but
% it should be fine as is for beginners.

% almost certainly you want these
\usepackage{amssymb}
\usepackage{amsmath}
\usepackage{amsfonts}

% used for TeXing text within eps files
%\usepackage{psfrag}
% need this for including graphics (\includegraphics)
%\usepackage{graphicx}
% for neatly defining theorems and propositions
%\usepackage{amsthm}
% making logically defined graphics
%%%\usepackage{xypic}

% there are many more packages, add them here as you need them

% define commands here
\usepackage{amsmath, amssymb, amsfonts, amsthm, amscd, latexsym,color,enumerate}
%%\usepackage{xypic}
\xyoption{curve}
\usepackage[mathscr]{eucal}

\setlength{\textwidth}{7.1in}
%\setlength{\textwidth}{16cm}
\setlength{\textheight}{9.2in}
%\setlength{\textheight}{24cm}

\hoffset=-1.0in     %%ps format
%\hoffset=-1.0in     %%hp format
\voffset=-.30in

%the next gives two direction arrows at the top of a 2 x 2 matrix

\newcommand{\directs}[2]{\def\objectstyle{\scriptstyle}  \objectmargin={0pt}
\xy
(0,4)*+{}="a",(0,-2)*+{\rule{0em}{1.5ex}#2}="b",(7,4)*+{\;#1}="c"
\ar@{->} "a";"b" \ar @{->}"a";"c" \endxy }

\theoremstyle{plain}
\newtheorem{lemma}{Lemma}[section]
\newtheorem{proposition}{Proposition}[section]
\newtheorem{theorem}{Theorem}[section]
\newtheorem{corollary}{Corollary}[section]
\newtheorem{conjecture}{Conjecture}[section]

\theoremstyle{definition}
\newtheorem{definition}{Definition}[section]
\newtheorem{example}{Example}[section]
%\theoremstyle{remark}
\newtheorem{remark}{Remark}[section]
\newtheorem*{notation}{Notation}
\newtheorem*{claim}{Claim}


\theoremstyle{plain}
\renewcommand{\thefootnote}{\ensuremath{\fnsymbol{footnote}}}
\numberwithin{equation}{section}
\newcommand{\Ad}{{\rm Ad}}
\newcommand{\Aut}{{\rm Aut}}
\newcommand{\Cl}{{\rm Cl}}
\newcommand{\Co}{{\rm Co}}
\newcommand{\DES}{{\rm DES}}
\newcommand{\Diff}{{\rm Diff}}
\newcommand{\Dom}{{\rm Dom}}
\newcommand{\Hol}{{\rm Hol}}
\newcommand{\Mon}{{\rm Mon}}
\newcommand{\Hom}{{\rm Hom}}
\newcommand{\Ker}{{\rm Ker}}
\newcommand{\Ind}{{\rm Ind}}
\newcommand{\IM}{{\rm Im}}
\newcommand{\Is}{{\rm Is}}
\newcommand{\ID}{{\rm id}}
\newcommand{\GL}{{\rm GL}}
\newcommand{\Iso}{{\rm Iso}}
\newcommand{\Sem}{{\rm Sem}}
\newcommand{\St}{{\rm St}}
\newcommand{\Sym}{{\rm Sym}}
\newcommand{\SU}{{\rm SU}}
\newcommand{\Tor}{{\rm Tor}}
\newcommand{\U}{{\rm U}}

\newcommand{\A}{\mathcal A}
\newcommand{\D}{\mathcal D}
\newcommand{\E}{\mathcal E}
\newcommand{\F}{\mathcal F}
\newcommand{\G}{\mathcal G}
\newcommand{\R}{\mathcal R}
\newcommand{\cS}{\mathcal S}
\newcommand{\cU}{\mathcal U}
\newcommand{\W}{\mathcal W}

\newcommand{\Ce}{\mathsf{C}}
\newcommand{\Q}{\mathsf{Q}}
\newcommand{\grp}{\mathsf{G}}
\newcommand{\dgrp}{\mathsf{D}}

\newcommand{\bA}{\mathbb{A}}
\newcommand{\bB}{\mathbb{B}}
\newcommand{\bC}{\mathbb{C}}
\newcommand{\bD}{\mathbb{D}}
\newcommand{\bE}{\mathbb{E}}
\newcommand{\bF}{\mathbb{F}}
\newcommand{\bG}{\mathbb{G}}
\newcommand{\bK}{\mathbb{K}}
\newcommand{\bM}{\mathbb{M}}
\newcommand{\bN}{\mathbb{N}}
\newcommand{\bO}{\mathbb{O}}
\newcommand{\bP}{\mathbb{P}}
\newcommand{\bR}{\mathbb{R}}
\newcommand{\bV}{\mathbb{V}}
\newcommand{\bZ}{\mathbb{Z}}

\newcommand{\bfE}{\mathbf{E}}
\newcommand{\bfX}{\mathbf{X}}
\newcommand{\bfY}{\mathbf{Y}}
\newcommand{\bfZ}{\mathbf{Z}}

\renewcommand{\O}{\Omega}
\renewcommand{\o}{\omega}
\newcommand{\vp}{\varphi}
\newcommand{\vep}{\varepsilon}

\newcommand{\diag}{{\rm diag}}
\newcommand{\desp}{{\mathbb D^{\rm{es}}}}
\newcommand{\Geod}{{\rm Geod}}
\newcommand{\geod}{{\rm geod}}
\newcommand{\hgr}{{\mathbb H}}
\newcommand{\mgr}{{\mathbb M}}
\newcommand{\ob}{\operatorname{Ob}}
\newcommand{\obg}{{\rm Ob(\mathbb G)}}
\newcommand{\obgp}{{\rm Ob(\mathbb G')}}
\newcommand{\obh}{{\rm Ob(\mathbb H)}}
\newcommand{\Osmooth}{{\Omega^{\infty}(X,*)}}
\newcommand{\ghomotop}{{\rho_2^{\square}}}
\newcommand{\gcalp}{{\mathbb G(\mathcal P)}}

\newcommand{\rf}{{R_{\mathcal F}}}
\newcommand{\glob}{{\rm glob}}
\newcommand{\loc}{{\rm loc}}
\newcommand{\TOP}{{\rm TOP}}

\newcommand{\wti}{\widetilde}
\newcommand{\what}{\widehat}

\renewcommand{\a}{\alpha}
\newcommand{\be}{\beta}
\newcommand{\ga}{\gamma}
\newcommand{\Ga}{\Gamma}
\newcommand{\de}{\delta}
\newcommand{\del}{\partial}
\newcommand{\ka}{\kappa}
\newcommand{\si}{\sigma}
\newcommand{\ta}{\tau}


\newcommand{\lra}{{\longrightarrow}}
\newcommand{\ra}{{\rightarrow}}
\newcommand{\rat}{{\rightarrowtail}}
\newcommand{\oset}[1]{\overset {#1}{\ra}}
\newcommand{\osetl}[1]{\overset {#1}{\lra}}
\newcommand{\hr}{{\hookrightarrow}}


\newcommand{\hdgb}{\boldsymbol{\rho}^\square}
\newcommand{\hdg}{\rho^\square_2}

\newcommand{\med}{\medbreak}
\newcommand{\medn}{\medbreak \noindent}
\newcommand{\bign}{\bigbreak \noindent}

\renewcommand{\leq}{{\leqslant}}
\renewcommand{\geq}{{\geqslant}}

\def\red{\textcolor{red}}
\def\magenta{\textcolor{magenta}}
\def\blue{\textcolor{blue}}
\def\<{\langle}
\def\>{\rangle}
\begin{document}
This is a topic entry on metric superfields in quantum supergravity
and the mathematical cncepts related to spinor and tensor fields.


\section {Metric superfields: spinor and tensor fields}

 Because in supergravity both spinor and tensor fields are being
considered, the gravitational fields are represented in terms of
\emph{tetrads}, $e^a_\mu(x),$ rather than in terms of the general
relativistic metric $g_{\mu \nu}(x)$. The connections between
these two distinct representations are as follows:
\begin{equation}
g_{\mu\nu}(x)=\eta_{ab}~ e^a_\mu (x)e^b_\gamma(x)~,
\end{equation}
with the general coordinates being indexed by $\mu,\nu,$ etc.,
whereas local coordinates that are being defined in a locally
inertial coordinate system are labeled with superscripts a, b,
etc.;   $ \eta_{ab}$ is the diagonal matrix with elements +1, +1,
+1 and -1. The tetrads are invariant to two distinct types of
symmetry transformations--the local Lorentz transformations:
\begin{equation}
e^a_\mu (x)\longmapsto \Lambda^a_b (x) e^b_\mu (x)~,
\end{equation}
(where $\Lambda^a_b$ is an arbitrary real matrix), and the general
coordinate transformations:
\begin{equation}
x^\mu \longmapsto (x')^\mu(x) ~.
\end{equation}
In a weak gravitational field the tetrad may be represented as:
\begin{equation}
e^a_\mu (x)=\delta^a_\mu(x)+ 2\kappa \Phi^a_\mu (x)~,
\end{equation}
where $\Phi^a_\mu(x)$ is small compared with $\delta^a_\mu(x)$ for
all $x$ values, and $\kappa= \surd 8\pi G$, where G is Newton's
gravitational constant. As it will be discussed next, the
supersymmetry algebra (SA) implies that the graviton has a
fermionic superpartner, the hypothetical \emph{gravitino}, with
helicities $\pm$ 3/2. Such a self-charge-conjugate massless
particle as the gravitiono with helicities $\pm$ 3/2 can only have
\emph{low-energy} interactions if it is represented by a Majorana
field $\psi _\mu(x)$ which is invariant under the gauge
transformations:
\begin{equation}
\psi _\mu(x)\longmapsto \psi _\mu(x)+\delta _\mu \psi(x) ~,
\end{equation}
with $\psi(x)$ being an arbitrary Majorana field as defined by
Grisaru and Pendleton (1977). The tetrad field $\Phi _{\mu
\nu}(x)$ and the graviton field $\psi _\mu(x)$ are then
incorporated into a term $H_\mu (x,\theta)$ defined as the
\emph{metric superfield}. The relationships between $\Phi _{\mu _
\nu}(x)$ and $\psi _\mu(x)$, on the one hand, and the components
of the metric superfield $H_\mu (x,\theta)$, on the other hand,
can be derived from the transformations of the whole metric
superfield:
\begin{equation}
H_\mu (x,\theta)\longmapsto H_\mu (x,\theta)+ \Delta _\mu
(x,\theta)~,
\end{equation}
by making the simplifying-- and physically realistic-- assumption
of a weak gravitational field (further details can be found, for
example, in Ch.31 of vol.3. of Weinberg, 1995). The interactions
of the entire superfield $H_\mu (x)$ with matter would be then
described by considering how a weak gravitational field,
$h_{\mu_\nu}$ interacts with an energy-momentum tensor $T^{\mu
\nu}$ represented as a linear combination of components of a real
vector superfield $\Theta^\mu$.  Such interaction terms would,
therefore, have the form:
\begin{equation}
 I_{\mathcal M}= 2\kappa \int dx^4 [H_\mu \Theta^\mu]_D ~,
\end{equation}
($\mathcal M$ denotes `matter') integrated over a four-dimensional
(Minkowski) spacetime with the metric defined by the superfield
$H_\mu (x,\theta)$. The term $\Theta^\mu$, as defined above, is
physically a \emph{supercurrent} and satisfies the conservation
conditions:
\begin{equation}
\gamma^\mu \mathbf{D} \Theta _\mu = \mathbf{D} ~,
\end{equation}
where $\mathbf{D}$ is the four-component super-derivative and $X$
denotes a real chiral scalar superfield. This leads immediately to
the calculation of the interactions of matter with a weak
gravitational field as:
\begin{equation}
I_{\mathcal M} = \kappa \int d^4 x T^{\mu \nu}(x)h_{\mu \nu}(x) ~,
\end{equation}
It is interesting to note that the gravitational actions for the
superfield that are invariant under the generalized gauge
transformations $H_\mu \longmapsto H _\mu  + \Delta _\mu$ lead to
solutions of the Einstein field equations for a homogeneous,
non-zero vacuum energy density $\rho _V$ that correspond to either
a de Sitter space for $\rho _V>0$, or an anti-de Sitter space for
$\rho _V <0$. Such spaces can be represented in terms of the
hypersurface equation
\begin{equation}
x^2_5 \pm \eta _{\mu,\nu} x^\mu x^\nu = R^2 ~,
\end{equation}
in a quasi-Euclidean five-dimensional space with the metric
specified as:
\begin{equation}
ds^2 = \eta _{\mu,\nu} x^\mu x^\nu \pm dx^2_5 ~,
\end{equation}
with '+' for de Sitter space and '-' for anti-de Sitter space,
respectively.

\textbf{Note}
The presentation above follows the exposition by S. Weinberg in his book
on ``Quantum Field Theory'' (2000), vol. 3, Cambridge University Press (UK),
in terms of both concepts and mathematical notations.

%%%%%
%%%%%
\end{document}
