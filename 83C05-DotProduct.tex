\documentclass[12pt]{article}
\usepackage{pmmeta}
\pmcanonicalname{DotProduct}
\pmcreated{2013-03-22 11:46:33}
\pmmodified{2013-03-22 11:46:33}
\pmowner{drini}{3}
\pmmodifier{drini}{3}
\pmtitle{dot product}
\pmrecord{13}{30239}
\pmprivacy{1}
\pmauthor{drini}{3}
\pmtype{Definition}
\pmcomment{trigger rebuild}
\pmclassification{msc}{83C05}
\pmclassification{msc}{15A63}
\pmclassification{msc}{14-02}
\pmclassification{msc}{14-01}
\pmsynonym{scalar product}{DotProduct}
\pmrelated{CauchySchwarzInequality}
\pmrelated{CrossProduct}
\pmrelated{Vector}
\pmrelated{DyadProduct}
\pmrelated{InvariantScalarProduct}
\pmrelated{AngleBetweenLineAndPlane}
\pmrelated{TripleScalarProduct}
\pmrelated{ProvingThalesTheoremWithVectors}
\pmdefines{scalar square}

\usepackage{amssymb}
\usepackage{amsmath}
\usepackage{amsfonts}
\usepackage{graphicx}
%%%%\usepackage{xypic}
\begin{document}
Let $u=(u_1,u_2,\ldots,u_n)$ and $v=(v_1,v_2,\ldots,v_n)$ two vectors on $k^n$ where $k$ is a field (like $\mathbb{R}$ or $\mathbb{C}$).
Then we define the \emph{dot product} of the two vectors as:
$$u\cdot v=u_1v_1+u_2v_2+\cdots+u_nv_n.$$

Notice that $u\cdot v$ is NOT a vector but a scalar (an element from the field $k$).

If $u,v$ are vectors in $\mathbb{R}^n$ and $\vartheta$ is the angle between them, then we also have
$$u\cdot v=\Vert u\Vert\Vert v\Vert \cos\vartheta.$$
Thus, in this case, $u\perp v$ if and only if $u\cdot v=0$.

The special case\, $u \cdot u$\, of scalar product is the {\em scalar square} of the vector $u$.\, In $\mathbb{R}^n$ it equals to the square of the length of $u$:
        $$u \cdot u = \Vert u \Vert^2$$
%%%%%
%%%%%
%%%%%
%%%%%
\end{document}
